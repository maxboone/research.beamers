\documentclass{beamer}
\usetheme{Madrid}
\usepackage{soul}


\title[CRDs for ngUML]{
    Using Kubernetes operators \& resource definitions
    for ngUML runtime generation 
}
\author{Max Boone}
\institute{LIACS}
\date{25th of May 2023}


\begin{document}

\frame{\titlepage}

\begin{frame}
    \frametitle{Contents}
    \tableofcontents
\end{frame} 

\section{Situation}

\begin{frame}{Current situation and background}
    ngUML end-to-end is ``prose(-to-metadata)-to-prototype". 
    This project focuses on how to get from metadata to prototype,
    in production.

    \begin{itemize}
        \item Runtime through code generation
        \item Generation by \texttt{str.format} templating
        \item Formatted and structured as Django Application (v3.1) 
        \item Stored in-tree / in modeler's Django Project
    \end{itemize}
\end{frame}

\section{Complications}

\begin{frame}{Complications}
    \begin{itemize}
        \item Same directory as main server directory: {
            \begin{itemize}
                \item Exceptions bubble up, killing entire system 
                \item Race conditions, as same thread is utilized
            \end{itemize}
        }
        \item Code generation to flat files: {
            \begin{itemize}
                \item Production environments' FS immutable
                \item Large risk for injection of malicious code
                \item Migration impractical, as entire system is overwritten
            \end{itemize} 
        }
    \end{itemize}
\end{frame}

\section{(Research) Question}

\begin{frame}{(Research) Question}
    Is a CRD-based approach a viable alternative to templated 
    code generation for the ngUML runtime?
\end{frame}

\section{Objectives}

\begin{frame}{Objectives}
    \begin{enumerate}
        \item Kubernetes Deployment
        \item Runtime CRDs \& Controller
        \item Django ORM construction without code generation
    \end{enumerate}
\end{frame}

\subsection{Kubernetes Deployment}

\begin{frame}{Objectives}{Kubernetes Deployment}
    Re-write the current \texttt{docker-compose.yaml} to a
    helm chart with a \texttt{values.yaml}. Add service
    account as well.

    ~

    Update \texttt{README.md} with instructions to helm 
    deploy locally (k3s, minikube, ...). Use \textbf{Gefyra}?
\end{frame}

\subsection{Runtime CRDs \& Controller}

\begin{frame}{Objectives}{Runtime CRDs \& Controller}
    \begin{block}{Kubernetes Documentation: Operator Pattern}
        Operators are software extensions to Kubernetes that make use of custom resources to manage applications and their components. Operators follow Kubernetes principles, notably the control loop.
    \end{block}

    ~

    Write CRDs for runtime metadata \& runtime configuration. 
    The custom controller should then build, upgrade or destroy a 
    runtime (with database) from this definition.
\end{frame}

\subsection{Django ORM construction}

\begin{frame}{Objectives}{Django ORM construction}
    Currently, the system generates Python code and stores it
    as files. We should be able to use the Django ORM without
    first rendering all metadata as Python code. We can find
    inspiration for this in \textbf{Baserow}: 
    the \textbf{SchemaEditor}.

    ~

    \begin{block}{Django Documentation: SchemaEditor}
        It's unlikely that you will want to interact directly with SchemaEditor as a normal developer using Django, but if you want to write your own migration system, or have more advanced needs, it's a lot nicer than writing SQL.
    \end{block}
\end{frame}

\section{Impact \& Risks}

\begin{frame}{Impact \& Risks}
    \begin{itemize}
        \item {
            \textbf{Helm Chart} \\
            Must ensure no degradation in DX (and 
            only required iff runtime needed). Will 
            impact code from all developers. 
        }
        \item {
            \textbf{Runtime CRDs \& Controller} \\
            Will be written in separate app (orchestrator).
        }
        \item {
            \textbf{Django ORM Construction} \\ 
            Will be written in separate app (runtime).
        }
    \end{itemize}
\end{frame}

%
% Dear other students, feel free to use the slides up till here
% for some structure. Next two slides are not necessary.
%

\begin{frame}{Something else}{Problem}
    Currently, the database structure is absurdly complex.
    We have 43-ish tables to store nodes in a diagram, 
    node type has it's own table \& django model.

    ~

    \begin{exampleblock}{Polymorphism - What is a chair?}
        Polymorphism, Jerry! It's like having a closet 
        full of clothes, but every time you reach for a 
        shirt, it magically turns into a pair of pants!
    \end{exampleblock}
    
    ~

    Django models don't support polymorphism, 
    so if you call \texttt{objects.all()} or 
    \texttt{objects.get(pk={\color{blue}id})} on 
    \texttt{Classifier} it does not return for 
    any models that inherit \texttt{Classifier}.
\end{frame}

\begin{frame}[fragile]{Something else}{What to do?}
    NoSQL to the rescue. We're not going to fully 
    implement the abstract structure in a relational 
    database. Instead: \texttt{JSONField}.

    ~

    However, no validation is {\color{red}dangerous}. Instead,
    we write JSON Schemas for nodes, extend Django ORM to 
    use \texttt{pg-jsonschema} and generate type definitions
    for Python \& TypeScript from the JSON Schemas. 

    ~

    Major change, as all code that directly calls the
    current classes will need changes. If you stick to 
    using the internal API, you should be fine. PR must 
    be non-breaking and requires review from 
    \textbf{everyone currently working}. Expected in two
    weeks.
\end{frame}

\end{document}